\documentclass[letterpaper,11pt]{article}
\usepackage[acronym]{glossaries}
\include{acros}
\usepackage{graphicx}
\usepackage{subfigure}
\graphicspath{ {images/} }
\usepackage{enumerate}
\usepackage{amssymb}
\usepackage{mathtools}
%\usepackage{minted}
\usepackage{tcolorbox}
\usepackage[margin=1in]{geometry}
\usepackage{caption}
\usepackage{float}
\usepackage{multirow}
\usepackage{hhline}
\usepackage{indentfirst}
\newcommand\numberthis{\addtocounter{equation}{1}\tag{\theequation}}
\renewcommand{\arraystretch}{1.2}
\linespread{1.3}


\begin{document}
%
\title{Implementation of the Spalart-Allmaras Turbulence Model in Moltres}
\author{Sun Myung Park}
\date{}
%
\maketitle
%
In fluid dynamics, turbulent flow is characterized by unsteady, irregular, and chaotic changes in
flow velocity as opposed to neat, parallel flow layers in laminar flow. Turbulent flows in
\glspl{MSR} induce significant turbulent diffusion effects on heat transfer and \gls{DNP} drift.
The resulting temperature and \gls{DNP} distributions strongly influence the neutron flux through
strong temperature reactivity feedback and delayed neutron emission in different neutron importance
regions, respectively. Flow separation and recirculation zones may also occur in \gls{MSR} designs
with significant wall curvature. Temperature hotspots occurring in recirculation zones may induce
thermal stresses on nearby structural components and cause temperature-induced reactivity effects.
In particular, the \gls{MCFR} will experience significant turbulent flow owing to its
pool-type \gls{MSR} design. When modeling the \gls{MCFR}, we require a turbulence model to
accurately model turbulent effects and associated multiphysics interactions. Otherwise, flow
modeling in coupled neutronics and thermal-hydraulics simulations of the \gls{MCFR} will be limited
to laminar flow. These simulations with laminar flow modeling will likely fail to capture the true
behavior of the \gls{MCFR} in steady-state operation and in transient accident scenarios.

The \gls{MSR} simulation tool we're employing to model the \gls{MCFR}, Moltres, does not
currently have turbulence modeling capability for simulating turbulent flows in
\glspl{MSR}. We propose implementing the \gls{SA} model, a one-equation turbulence model,
in Moltres to address this issue. Since Moltres is built on the \gls{MOOSE} framework, we can
leverage the advanced and developer-friendly features in \gls{MOOSE} to implement an \gls{SA} model
that is highly scalable and compatible with the existing fluid dynamics modeling infrastructure in
\gls{MOOSE}. Once implemented, we will verify and validate the model against published data for the
backward-facing step, a common reference problem for turbulence modeling. This work and the
subsequent demonstration of the \gls{SA} model for turbulent flow modeling in a \gls{MCFR} model
will be submitted for conference publication.

\end{document}
